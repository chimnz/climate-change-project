\documentclass[12pt]{article}
\usepackage[margin=1.0in]{geometry}
\usepackage{graphicx}
\usepackage{amsmath} % needed for "boxed" command and bmatrix
\usepackage[makeroom]{cancel} % needed for "cancel" command
\usepackage{multicol}
\usepackage{parskip} % removes paragraph indents
\usepackage{listings}
\usepackage{setspace}
\usepackage{hyperref}

% \pagenumbering{gobble}

\newcommand{\img}[4]{
 \begin{center}
 \includegraphics[scale=#1]{#2} \\
 \textbf{Figure #3.} #4
 \end{center}
}

\begin{document}

{\LARGE Climate Change Project}

{\large Christian Chimezie}

{\large ATM 320}

\noindent\makebox[\linewidth]{\rule{16.51cm}{0.4pt}}

\section*{Introduction}
The purpose of this project is to determine whether climate-change is
happening on a global scale by analyzing atmospheric data recorded
over the last 40 year period.
Global temperature is represented in terms of an anomaly,
which is a departure from a long-term average.
The Southern Oscillation Index (SOI) is an index based on observed
sea-level pressure differences between Tahita and Austria;
this index might be useful in modeling atmospheric temperature
trends.
We will analyze trends in global temperature anomaly, SOI, and also
local temperatures in Albany, New York.
We hope that analyzing historical atmospheric data will give insight
into the phenomenon of climate change.

\section*{Temperature and SOI Statistics}
\subsection*{1980 - 1999}
Local temperature:
\begin{center}
\begin{tabular}{c | c | c | c}
 $T_{avg} \textrm{ } (^{\circ} \textrm{C})$ &
 $\sigma T \textrm{ } (^{\circ} \textrm{C})$ &
 $T_{max} \textrm{ } (^{\circ} \textrm{C})$ &
 $T_{min} \textrm{ } (^{\circ} \textrm{C})$ \\ %%
 \hline
 8.58 & 10.48 & 22.4 & -10.0 \\ %%
\end{tabular}
\end{center}
Global temperature anomaly:
\begin{center}
\begin{tabular}{c | c | c | c}
 $T_{avg} \textrm{ } (^{\circ} \textrm{C})$ &
 $\sigma T \textrm{ } (^{\circ} \textrm{C})$ &
 $T_{max} \textrm{ } (^{\circ} \textrm{C})$ &
 $T_{min} \textrm{ } (^{\circ} \textrm{C})$ \\ %%
 \hline
 0.3 & 0.1 & 0.52 & 0.13 \\ %%
\end{tabular}
\end{center}
Southern Oscillation Index:
\begin{center}
\begin{tabular}{c | c | c | c}
 avg & $\sigma$ & max & min \\ %%
 \hline
 0.12 & 0.93 & 2.0 & -2.1 \\ %%
\end{tabular}
\end{center}

\subsection*{2000 - 2019}
Local temperature:
\begin{center}
\begin{tabular}{c | c | c | c}
 $T_{avg} \textrm{ } (^{\circ} \textrm{C})$ &
 $\sigma T \textrm{ } (^{\circ} \textrm{C})$ &
 $T_{max} \textrm{ } (^{\circ} \textrm{C})$ &
 $T_{min} \textrm{ } (^{\circ} \textrm{C})$ \\ %%
 \hline
 9.24 & 9.6 & 24.7 & -10.7 \\ %%
\end{tabular}
\end{center}
Global temperature anomaly:
\begin{center}
\begin{tabular}{c | c | c | c}
 $T_{avg} \textrm{ } (^{\circ} \textrm{C})$ &
 $\sigma T \textrm{ } (^{\circ} \textrm{C})$ &
 $T_{max} \textrm{ } (^{\circ} \textrm{C})$ &
 $T_{min} \textrm{ } (^{\circ} \textrm{C})$ \\ %%
 \hline
 0.52 & 0.26 & 1.37 & -0.04 \\ %%
\end{tabular}
\end{center}
Southern Oscillation Index:
\begin{center}
\begin{tabular}{c | c | c | c}
 avg & $\sigma$ & max & min \\ %%
 \hline
 0.03 & 1.62 & 4.8 & -6.0 \\ %%
\end{tabular}
\end{center}

\subsection*{1980 - 2019}
Local temperature:
\begin{center}
\begin{tabular}{c | c | c | c}
 $T_{avg} \textrm{ } (^{\circ} \textrm{C})$ &
 $\sigma T \textrm{ } (^{\circ} \textrm{C})$ &
 $T_{max} \textrm{ } (^{\circ} \textrm{C})$ &
 $T_{min} \textrm{ } (^{\circ} \textrm{C})$ \\ %%
 \hline
 9.21 & 9.64 & 24.7 & -10.7 \\ %%
\end{tabular}
\end{center}
Global temperature anomaly:
\begin{center}
\begin{tabular}{c | c | c | c}
 $T_{avg} \textrm{ } (^{\circ} \textrm{C})$ &
 $\sigma T \textrm{ } (^{\circ} \textrm{C})$ &
 $T_{max} \textrm{ } (^{\circ} \textrm{C})$ &
 $T_{min} \textrm{ } (^{\circ} \textrm{C})$ \\ %%
 \hline
 0.51 & 0.26 & 1.37 & -0.04 \\ %%
\end{tabular}
\end{center}
Southern Oscillation Index:
\begin{center}
\begin{tabular}{c | c | c | c}
 avg & $\sigma$ & max & min \\ %%
 \hline
 0.04 & 1.6 & 4.8 & -6.0 \\ %%
\end{tabular}
\end{center}

\section*{Visualizing Temperature and SOI Trends}
% best fit lines
\img{0.65}{../plots/fits/local.png}{1}{
 Degree 1 polynomial fit of local temperature as a
 function of time.
}
\img{0.65}{../plots/fits/global.png}{2}{
 Degree 1 polynomial fit of global temperature anomaly as a
 function of time.
}
\img{0.65}{../plots/fits/soi.png}{3}{
 Degree 1 polynomial fit of SOI as a
 function of time.
}
% comparisons
\img{0.65}{../plots/compare/local_vs_global.png}{4}{
 Degree 1 polynomial fit of local temperature compared
 to global temperatue anomaly.
}
\img{0.65}{../plots/compare/local_vs_soi.png}{5}{
 Degree 1 polynomial fit of local temperature compared
 to SOI.
}
% histograms
\img{0.65}{../plots/histogram/local.png}{6}{
 Histogram of monthly local temperature values.
}
\img{0.65}{../plots/histogram/global.png}{7}{
 Histogram of monthly global temperature anomaly values.
}
\img{0.65}{../plots/histogram/soi.png}{8}{
<<<<<<< HEAD
 Histogram of monthly SOI values.
=======
<<<<<<< HEAD
 Histogram of monthly SOI values.
=======
 Histogram of SOI values.
>>>>>>> 607bfba2ab0830c6dbadc95c9d5db15e38fb4158
>>>>>>> 8a10b3323ca51b1f8a4511463229ac5de8bc45ac
}

\section*{Local and Global Temperature Correlation}
The correlation coefficient is a quantification of the
strength of the linear relationship between two variables.
Correlations of -1 or +1 imply an exact linear relationship,
while 0 implies no correlation.
The Pearson correlation coefficient is calculated using
the equation \ref{eq1}.

\begin{equation}\label{eq1}
 r = \frac{\sum (x - \overline{x}) (y - \overline{y})}
 {\sqrt{\sum (x - \overline{x})^2 \sum (y - \overline{y})^2}}
\end{equation}

The statistical significance of the correlation coefficient can be
interpreted using the p-value method.
The p-value $P$ is calculated using a t-distribution with
$n - 2$ degrees of freedom as shown in equation \ref{eq2}.

\begin{equation}\label{eq2}
 t = \frac{r \sqrt{n-2}}{\sqrt{1 - r^2}}
\end{equation}

We use $\alpha = 0.05$ as the threshold significance level.
\begin{itemize}
 \item If $P < \alpha$, the variables \textit{are}
 linear because $r$ \textit{is} sufficiently different from zero.
 \item If $P > \alpha$, the variables \textit{are not}
 linear because $r$ \textit{is not} sufficiently different from zero.
\end{itemize}

We calculate a correlation coefficient of $r = 0.673$ for
the local temperature and global temperature anomaly,
and the corresponding p-value is $P = 1.94 \times 10^{-6}$.
The p-value is far below the threshold, so we conclude that
there is a correlation between local temperature and
global temperature anomaly.

For local temperature and SOI, we calculate a correlation
coefficient of $r = 0.171$ and a p-value of $P = 0.291$.
<<<<<<< HEAD
The p-value is larger than the threshold, so we
=======
The p-value is larger than the threshold hold, so we
>>>>>>> 607bfba2ab0830c6dbadc95c9d5db15e38fb4158
conclude that there is no correlation between local temperature
and the Southern Oscillation Index.

\section*{Analysis of Local Temperature Trends}
The question of whether the climate is getting warmer at the
location in question, Albany, New York, is answered by
the trend in figure 1.
From 1980 to 2019, the local temperature has steadily increased,
as indicated by the upward trend illustrated by the best-fit line.

But is this local warming related to global warming?
To answer this question we must first look at trends in
global temperature anomaly.
The strong upward trend in figure 2 tells us global temperature is
steadily increasing.
Then, we look to our calculation of the correlation coefficient of
for local temperature and global temperature anomaly;
our calculation indicates that local temperature is directly
correlated with global temperature anomaly.
Therefore, the data proves that local warming is due to in part
to global warming.

El Ni\~{n}o-Southern Oscillation (ENSO), the periodic fluctuation in
surface sea temperatures over the Pacific Ocean, is likely not
the cause of this local climate variation.
This is because the ENSO fluctuation occurs over a 2-7 year timescale,
and the local temperature has risen over a 40 year time period.
In addition to this, Albany, New York is landlocked, and it is not
sufficiently close to the Pacific Ocean such that surface sea
temperature fluctuations would significantly affect its climate.

\section*{Conclusion}
After doing an extensive analysis of global temperature trends,
we have concluded that global climate change is undoubtedly a
real thing.
We analyzed the climate of a particular location over a
40 year period and concluded that the upward temperature trend
is due to global warming and is not perpetrated by ENSO.
Though there will always be naysayers, society as a whole must
face the facts: the earth is getting warmer.
Making predictions on the implications of this global
temperature rise is beyond the scope of this project, but this is
a very important topic to explore since climate-change will affect
every single living being on this planet regardless of
personal beliefs or political affiliation.
So, it is imperative that further research into the effects of
climate-change is not hampered by those who wish to pretend that
this aforementioned change is not occurring when the data
clearly states otherwise.

\end{document}