\documentclass[12pt]{article}
\usepackage[margin=1.0in]{geometry}
\usepackage{graphicx}
\usepackage{amsmath} % needed for "boxed" command and bmatrix
\usepackage[makeroom]{cancel} % needed for "cancel" command
\usepackage{multicol}
\usepackage{parskip} % removes paragraph indents
\usepackage{listings}
\usepackage{setspace}
\usepackage{hyperref}

% \pagenumbering{gobble}

\newcommand{\img}[4]{
 \begin{center}
 \includegraphics[scale=#1]{#2} \\
 \textbf{Figure #3.} #4
 \end{center}
}

\begin{document}

{\LARGE Climate Change Project}

{\large Christian Chimezie}

{\large ATM 320}

\noindent\makebox[\linewidth]{\rule{16.51cm}{0.4pt}}

\section*{Tables}

\subsection*{1980 - 1999}
Local temperature:
\begin{center}
\begin{tabular}{c | c | c | c}
	$T_{avg} \textrm{ } (^{\circ} \textrm{F})$ &
	$\sigma T \textrm{ } (^{\circ} \textrm{F})$ &
	$T_{max} \textrm{ } (^{\circ} \textrm{F})$ &
	$T_{min} \textrm{ } (^{\circ} \textrm{F})$ \\ %%
	\hline
	47.435 & 18.869 & 72.2 & 14.0 \\ %%
\end{tabular}
\end{center}
Global temperature:
\begin{center}
\begin{tabular}{c | c | c | c}
	$T_{avg} \textrm{ } (^{\circ} \textrm{F})$ &
	$\sigma T \textrm{ } (^{\circ} \textrm{F})$ &
	$T_{max} \textrm{ } (^{\circ} \textrm{F})$ &
	$T_{min} \textrm{ } (^{\circ} \textrm{F})$ \\ %%
	\hline
	0.302 & 0.098 & 0.52 & 0.13 \\ %%
\end{tabular}
\end{center}
SOI index:
\begin{center}
\begin{tabular}{c | c | c | c}
	avg & $\sigma$ & max & min \\ %%
	\hline
	0.12 & 0.93 & 2.0 & -2.1 \\ %%
\end{tabular}
\end{center}

\subsection*{2000 - 2019}
Local temperature:
\begin{center}
\begin{tabular}{c | c | c | c}
	$T_{avg} \textrm{ } (^{\circ} \textrm{F})$ &
	$\sigma T \textrm{ } (^{\circ} \textrm{F})$ &
	$T_{max} \textrm{ } (^{\circ} \textrm{F})$ &
	$T_{min} \textrm{ } (^{\circ} \textrm{F})$ \\ %%
	\hline
	48.621 & 17.288 & 76.5 & 12.7 \\ %%
\end{tabular}
\end{center}
Global temperature:
\begin{center}
\begin{tabular}{c | c | c | c}
	$T_{avg} \textrm{ } (^{\circ} \textrm{F})$ &
	$\sigma T \textrm{ } (^{\circ} \textrm{F})$ &
	$T_{max} \textrm{ } (^{\circ} \textrm{F})$ &
	$T_{min} \textrm{ } (^{\circ} \textrm{F})$ \\ %%
	\hline
	0.516 & 0.262 & 1.37 & -0.04 \\ %%
\end{tabular}
\end{center}
SOI index:
\begin{center}
\begin{tabular}{c | c | c | c}
	avg & $\sigma$ & max & min \\ %%
	\hline
	0.034 & 1.621 & 4.8 & -6.0 \\ %%
\end{tabular}
\end{center}

\subsection*{1980 - 2019}
Local temperature:
\begin{center}
\begin{tabular}{c | c | c | c}
	$T_{avg} \textrm{ } (^{\circ} \textrm{F})$ &
	$\sigma T \textrm{ } (^{\circ} \textrm{F})$ &
	$T_{max} \textrm{ } (^{\circ} \textrm{F})$ &
	$T_{min} \textrm{ } (^{\circ} \textrm{F})$ \\ %%
	\hline
	48.571 & 17.359 & 76.5 & 12.7 \\ %%
\end{tabular}
\end{center}
Global temperature:
\begin{center}
\begin{tabular}{c | c | c | c}
	$T_{avg} \textrm{ } (^{\circ} \textrm{F})$ &
	$\sigma T \textrm{ } (^{\circ} \textrm{F})$ &
	$T_{max} \textrm{ } (^{\circ} \textrm{F})$ &
	$T_{min} \textrm{ } (^{\circ} \textrm{F})$ \\ %%
	\hline
	0.507 & 0.261 & 1.37 & -0.04 \\ %%
\end{tabular}
\end{center}
SOI index:
\begin{center}
\begin{tabular}{c | c | c | c}
	avg & $\sigma$ & max & min \\ %%
	\hline
	0.038 & 1.598 & 4.8 & -6.0 \\ %%
\end{tabular}
\end{center}

\section*{Correlation Coefficient}
The correlation coefficient is a quantification of the
strength of the linear relationship between two variables.
Correlations of -1 or +1 imply an exact linear relationship,
while 0 implies no correlation.
The Pearson correlation coefficient is calculated using
the equation \ref{eq1}.

\begin{equation}\label{eq1}
	r = \frac{\sum (x - \overline{x}) (y - \overline{y})}
        {\sqrt{\sum (x - \overline{x})^2 \sum (y - \overline{y})^2}}
\end{equation}

The statistical significance of the correlation coefficient can be
interpreted using the p-value method.
The p-value $P$ is calculated using a t-distribution with
$n - 2$ degrees of freedom as shown in equation \ref{eq2}.

\begin{equation}\label{eq2}
	t = \frac{r \sqrt{n-2}}{\sqrt{1 - r^2}}
\end{equation}

We use $\alpha = 0.05$ as the threshold significance level.
\begin{itemize}
	\item If $P < \alpha$, the variables \textit{are}
	linear because $r$ \textit{is} sufficiently different from zero.
	\item If $P > \alpha$, the variables \textit{are not}
	linear because $r$ \textit{is not} sufficiently different from zero.
\end{itemize}

\end{document}