\documentclass[12pt]{article}
\usepackage[margin=1.0in]{geometry}
\usepackage{graphicx}
\usepackage{amsmath} % needed for "boxed" command and bmatrix
\usepackage[makeroom]{cancel} % needed for "cancel" command
\usepackage{multicol}
\usepackage{parskip} % removes paragraph indents
\usepackage{listings}
\usepackage{setspace}
\usepackage{hyperref}

% \pagenumbering{gobble}

\newcommand{\img}[4]{
 \begin{center}
 \includegraphics[scale=#1]{#2} \\
 \textbf{Figure #3.} #4
 \end{center}
}

\begin{document}

{\LARGE Climate Change Project}

{\large Christian Chimezie}

{\large ATM 320}

\noindent\makebox[\linewidth]{\rule{16.51cm}{0.4pt}}

\section*{Introduction}
The purpose of this project is to determine whether climate-change is
happening on a global scale by analyzing atmospheric data recorded
over the last 40 year period.
Global temperatures are represented in terms of anomalies,
which are departures from a long-term average.
The Southern Oscillation Index (SOI) is an index based on observed
sea-level pressure differences between Tahita and Austria;
this index might be useful in modeling atmospheric temperature
trends.
We will analyze trends in global temperature anomaly, SOI, and also
local temperatures in Albany, New York.
We hope that analyzing historical atmospheric data will give insight
into the phenomenon of climate change.

\section*{Temperature and SOI Statistics}
\subsection*{1980 - 1999}
Local temperature:
\begin{center}
\begin{tabular}{c | c | c | c}
	$T_{avg} \textrm{ } (^{\circ} \textrm{C})$ &
	$\sigma T \textrm{ } (^{\circ} \textrm{C})$ &
	$T_{max} \textrm{ } (^{\circ} \textrm{C})$ &
	$T_{min} \textrm{ } (^{\circ} \textrm{C})$ \\ %%
	\hline
	8.58 & 10.48 & 22.4 & -10.0 \\ %%
\end{tabular}
\end{center}
Global temperature anomaly:
\begin{center}
\begin{tabular}{c | c | c | c}
	$T_{avg} \textrm{ } (^{\circ} \textrm{C})$ &
	$\sigma T \textrm{ } (^{\circ} \textrm{C})$ &
	$T_{max} \textrm{ } (^{\circ} \textrm{C})$ &
	$T_{min} \textrm{ } (^{\circ} \textrm{C})$ \\ %%
	\hline
	0.3 & 0.1 & 0.52 & 0.13 \\ %%
\end{tabular}
\end{center}
Southern Oscillation Index:
\begin{center}
\begin{tabular}{c | c | c | c}
	avg & $\sigma$ & max & min \\ %%
	\hline
	0.12 & 0.93 & 2.0 & -2.1 \\ %%
\end{tabular}
\end{center}

\subsection*{2000 - 2019}
Local temperature:
\begin{center}
\begin{tabular}{c | c | c | c}
	$T_{avg} \textrm{ } (^{\circ} \textrm{C})$ &
	$\sigma T \textrm{ } (^{\circ} \textrm{C})$ &
	$T_{max} \textrm{ } (^{\circ} \textrm{C})$ &
	$T_{min} \textrm{ } (^{\circ} \textrm{C})$ \\ %%
	\hline
	9.24 & 9.6 & 24.7 & -10.7 \\ %%
\end{tabular}
\end{center}
Global temperature anomaly:
\begin{center}
\begin{tabular}{c | c | c | c}
	$T_{avg} \textrm{ } (^{\circ} \textrm{C})$ &
	$\sigma T \textrm{ } (^{\circ} \textrm{C})$ &
	$T_{max} \textrm{ } (^{\circ} \textrm{C})$ &
	$T_{min} \textrm{ } (^{\circ} \textrm{C})$ \\ %%
	\hline
	0.52 & 0.26 & 1.37 & -0.04 \\ %%
\end{tabular}
\end{center}
Southern Oscillation Index:
\begin{center}
\begin{tabular}{c | c | c | c}
	avg & $\sigma$ & max & min \\ %%
	\hline
	0.03 & 1.62 & 4.8 & -6.0 \\ %%
\end{tabular}
\end{center}

\subsection*{1980 - 2019}
Local temperature:
\begin{center}
\begin{tabular}{c | c | c | c}
	$T_{avg} \textrm{ } (^{\circ} \textrm{C})$ &
	$\sigma T \textrm{ } (^{\circ} \textrm{C})$ &
	$T_{max} \textrm{ } (^{\circ} \textrm{C})$ &
	$T_{min} \textrm{ } (^{\circ} \textrm{C})$ \\ %%
	\hline
	9.21 & 9.64 & 24.7 & -10.7 \\ %%
\end{tabular}
\end{center}
Global temperature anomaly:
\begin{center}
\begin{tabular}{c | c | c | c}
	$T_{avg} \textrm{ } (^{\circ} \textrm{C})$ &
	$\sigma T \textrm{ } (^{\circ} \textrm{C})$ &
	$T_{max} \textrm{ } (^{\circ} \textrm{C})$ &
	$T_{min} \textrm{ } (^{\circ} \textrm{C})$ \\ %%
	\hline
	0.51 & 0.26 & 1.37 & -0.04 \\ %%
\end{tabular}
\end{center}
Southern Oscillation Index:
\begin{center}
\begin{tabular}{c | c | c | c}
	avg & $\sigma$ & max & min \\ %%
	\hline
	0.04 & 1.6 & 4.8 & -6.0 \\ %%
\end{tabular}
\end{center}

\section*{Visualizing Temperature and SOI Trends}
% best fit lines
\img{0.65}{../plots/fits/local.png}{1}{
	Degree 1 polynomial fit of local temperature as a
	function of time.
}
\img{0.65}{../plots/fits/global.png}{2}{
	Degree 1 polynomial fit of global temperature anomaly as a
	function of time.
}
\img{0.65}{../plots/fits/soi.png}{3}{
	Degree 1 polynomial fit of SOI as a
	function of time.
}
% comparisons
\img{0.65}{../plots/compare/local_vs_global.png}{4}{
	Degree 1 polynomial fit of local temperature compared
	to global temperatue anomaly.
}
\img{0.65}{../plots/compare/local_vs_soi.png}{5}{
	Degree 1 polynomial fit of local temperature compared
	to SOI.
}
% histograms
\img{0.65}{../plots/histogram/local.png}{6}{
	Histogram of monthly local temperature values.
}
\img{0.65}{../plots/histogram/global.png}{6}{
	Histogram of monthly global temperature anomaly values.
}
\img{0.65}{../plots/histogram/soi.png}{6}{
	Histogram of SOI values.
}

\section*{Correlation Coefficient}
The correlation coefficient is a quantification of the
strength of the linear relationship between two variables.
Correlations of -1 or +1 imply an exact linear relationship,
while 0 implies no correlation.
The Pearson correlation coefficient is calculated using
the equation \ref{eq1}.

\begin{equation}\label{eq1}
	r = \frac{\sum (x - \overline{x}) (y - \overline{y})}
        {\sqrt{\sum (x - \overline{x})^2 \sum (y - \overline{y})^2}}
\end{equation}

The statistical significance of the correlation coefficient can be
interpreted using the p-value method.
The p-value $P$ is calculated using a t-distribution with
$n - 2$ degrees of freedom as shown in equation \ref{eq2}.

\begin{equation}\label{eq2}
	t = \frac{r \sqrt{n-2}}{\sqrt{1 - r^2}}
\end{equation}

We use $\alpha = 0.05$ as the threshold significance level.
\begin{itemize}
	\item If $P < \alpha$, the variables \textit{are}
	linear because $r$ \textit{is} sufficiently different from zero.
	\item If $P > \alpha$, the variables \textit{are not}
	linear because $r$ \textit{is not} sufficiently different from zero.
\end{itemize}

\end{document}